% !TEX program = xelatex

\documentclass{resume}
% \usepackage{zh_CN-Adobefonts_external} % Simplified Chinese Support using external fonts (./fonts/zh_CN-Adobe/)
% \usepackage{zh_CN-Adobefonts_internal} % Simplified Chinese Support using system fonts
\usepackage{fontawesome5}
\usepackage[AutoFakeBold=true, AutoFakeSlant=true]{xeCJK}

\setCJKfamilyfont{Songti SC}[AutoFakeBold = {2.0}]{Songti SC}
\newCJKfontfamily{\STHeiti}{STHeiti}
\newCJKfontfamily{\STKaiti}{STKaiti}

\renewcommand{\baselinestretch}{1.1}
\hypersetup{hidelinks}

\begin{document}
\pagenumbering{gobble} % suppress displaying page number

\name{\STHeiti{梁莉莉(Lili Liang)}}

\basicInfo{
    (86) 175-4399-9485
    | \href{mailto:l2liang@cmu.edu}{l2liang@cmu.edu} 
    | \href{https://github.com/leungll}{\faGithub \ leungll} 
    | \href{https://www.linkedin.com/in/l2liang}{\faLinkedin \ l2liang}
    | \href{https://leungll.site/about}{\faAngleRight \ leungll.site}
}

\section{教育经历}
\datedsubsection{\textbf{\href{https://www.cmu.edu}{Carnegie Mellon University(卡内基梅隆大学)}}}{美国, 加利福尼亚州}
\datedsubsection{硕士,\textit{Master of Science in Software Engineering}}{2024.08 - 至今}

\datedsubsection{\textbf{\href{https://www.nenu.edu.cn}{东北师范大学}}}{吉林, 长春}
\datedsubsection{本科,软件工程,均分: 88.16/100,排名:12/108(11\%)}{2017.09 -- 2021.06}

% \section{论文发表}
% \begin{itemize}
%   \item Junping Zhou, Chumin Li, Yupeng Zhou, Mingyang Li, \textbf{Lili Liang}, and Jianan Wang, 
%   ``Solving diversified top-k weight clique search problem'', 
%   in \textbf{\href{https://link.springer.com/article/10.1007/s11432-020-3069-4}{\textit{Science China Information Sciences}}} 
%   and \textbf{\href{https://hsi-workshop.github.io/hsi2020-website/program.html}{\textit{HSI 2020(conjunction with IJCAI 2020)}}}, 
%   \href{https://hsi-workshop.github.io/hsi2020-website/hsi2020/Solving%20diversified%20top-k%20weight%20clique%20search%20problem%20with%20MaxSAT.pdf}{[PDF]}
% \end{itemize}

\section{技术能力}
\textbf{编程语言}:Golang, Java, C/C++, SQL, Python, JavaScript, HTML/CSS, Markdown

\textbf{技术框架}:RPC(Thrift), RocketMQ, Kafka, SpringBoot, MyBatis, Node.js, Zookeeper

\textbf{数据库}:MySQL, Redis, ElasticSearch, Hive, MongoDB

\textbf{工具与服务}:Linux, Git, \LaTeX, Swagger, Google Cloud Platform, Amazon Web Services

\section{工作经历}
\datedsubsection{\href{https://www.bytedance.com}{\textbf{字节跳动} (TikTok、抖音母公司)}}{广东, 深圳}
\datedsubsection{\STKaiti {后台研发工程师,全职,TikTok 电商履约中台}}{2021.07 -- 2023.10}
\begin{itemize}[parsep=0.3ex] \normalsize
    \item \textbf{商家履约}:核心研发,支持商家履约多端能力建设,先后参与多个MVP项目、大型横向项目开发,业务日均订单量由5.6w增长到1000w。
    \item \textbf{OpenAPI}:业务Owner,梳理历史架构、跟踪线上问题,发现16处历史bug并及时响应进行修复,完成多个系统能力优化和需
    求上线牵头制定接口变更规范,为ISV提供丰富的OpenAPI履约能力。
    \item \textbf{稳定性建设}:业务Owner,负责业务问题排查工具、成功率看板的建设,牵头完成全链路工具上报SDK、数据清洗、全场景数
    据看板等基建。对于线上问题排查,节省75\%人力资源(21人缩减至5人)。
    \item \textbf{工作成绩}:“超出预期”绩效晋升 (top 1\%),国际电商 \textit{Spot Bonus} (突出工作表现,top 3\%)
\end{itemize}

\section{项目经历}
\datedsubsection{\textbf{履约决策系统与配置SDK项目 @字节跳动}}{2022.09 -- 2022.11}
\datedsubsection{\STKaiti {项目负责人,技术:Golang, KiteX, RocketMQ, SDK, RPC, Metrics, Grafana}}{}
\begin{itemize}[parsep=0.3ex] \normalsize
    \item 实现一个能够封装业务决策逻辑、实现可配置、支持灰度发布机制和异常回滚的决策系统。
    \item 在MVP版本中,将25条业务规则转化为规则表达式,设计规则引擎完成正则表达式的逻辑计算。
    \item 使用公司基础组件TCC进行规则配置,提供规则管理及读写能力,完成业务规则的配置管理。
    \item 设计动作校验服务,分别提供SDK、RPC两种接入方式,避免单点问题的出现。
    \item \textbf{项目成果}:
      \begin{itemize}
        \item[$\circ$] 收敛商家履约业务决策逻辑,支持未来新规则的低成本接入
        \item[$\circ$] 灰度上线3个月,接入决策SDK的QPS:1.1k,接入决策系统RPC服务的QPS:115(B端业务)
      \end{itemize}
\end{itemize}

\section{研究经历}
\datedsubsection{论文:\href{https://link.springer.com/article/10.1007/s11432-020-3069-4}{\textbf{Solving Diversified 
Top-k Weight Clique Search Problem}}}{2020.07 -- 2020.09}
\datedsubsection{\STKaiti {研究方向:算法求解}}{}
\begin{itemize}[parsep=0.3ex] \normalsize
    \item 文章提出2种解决多样化top-k加权团搜索(DTKWCS)问题的编码策略以及DTKWCS的2个具体实际应用。
    \item 工作职能:
        \begin{itemize}
          \item[$\circ$] 独立实现论文直接编码、基于独立集划分的2种求解方式;
          \item[$\circ$] 提出一种针对顶点拓展数量,进行对称压缩破坏的处理方法,提高软子句的求解效率。
        \end{itemize}
    \item 发表于JCR Q1区期刊:\textbf{\href{http://scis.scichina.com}{\textit{Science China Information Sciences}}}, 
    会议:\textbf{\href{https://hsi-workshop.github.io/hsi2020-website/program.html}{\textit{HSI 2020(conjunction with IJCAI 2020)}}}
\end{itemize}

\section{荣誉奖励}
\begin{itemize}[parsep=0.2ex] \normalsize
    \item 字节跳动,“超出预期”绩效晋升 (top 1\%) \hfill 2023
    \item 字节跳动,TikTok国际电商 \textit{Spot Bonus} (突出工作表现,top 3\%) \hfill 2022
    \item 东北师范大学一等奖学金、东北师范大学优秀学生、实践创新类奖学金(top 7\%)  \hfill 2021
    \item 东北师范大学校长奖学金、东北师范大学优秀学生、年度“创新之星” (top 3\%)  \hfill 2020
    \item 东北师范大学二等奖学金、东北师范大学优秀学生  \hfill 2019
    \item (国家级)“全国高校绿色计算大赛”全国一等奖、团体一等奖(第1参加人)\hfill 2018
\end{itemize}

\section{技术博客}
\begin{itemize}[parsep=0.3ex] \normalsize
    \item \href{https://blog.csdn.net/liangllhahaha?type=blog}{[\faArrowRight \ CSDN]}:输出\textbf{60+}篇算法、工程类技术博客, 
    获得\textbf{59w+} 访问量;

    \item \href{https://github.com/leungll}{[\faGithub \ GitHub]}:
    \href{https://github.com/leungll/NENU-Courses}{\textbf{NENU-Courses}} 课程攻略开源项目发起者;

    \item \href{https://leungll.site/about}{[\faArrowRight \ leungll.site]}:独自搭建的个人站点,站点双线部署于GitHub、Gitee;
    基于JsDelivr搭建免费CDN,优化静态资源加载速度;应用站点流量分析工具;提交搜索引擎收录(SEO)。
\end{itemize}

\end{document}
